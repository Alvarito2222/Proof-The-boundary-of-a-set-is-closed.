\documentclass{article}
\usepackage[utf8]{inputenc}
\usepackage{amssymb} 
\usepackage{parskip} 

\title{Proof : The intersection of two open sets is open}
\author{aleonp000 }
\date{January 2023}

\begin{document}


\noindent
\textbf{Theorem}. \textit{10. The boundary of a set is closed. }

\textit{Def: } A point $p$, is a boundary point of the set $D$ , if $\forall$ $nbhd$ $N$ of 
$p$
 , $N$ contains points of both $D$ and $D^{c}$. 


\textit{Proof: }
 Let $D$ be a point set. Suppose that $bd(D)$ is not closed.
 This means that there is some limit point $l$ of $bd(D)$ $s.t.$ $l \notin$$bd(D)$. 
 
Since $l$ is a limit point of $bd(D)$, $\forall$ $nhbd$ $N$ of $l$ , $\exists$ a point $q \in$ $bd(D)$ $\cap$$N$ $\setminus\{l\}$ . $N$ is also a $nhbd$ of $q \in bd(D)$. 

We have shown $\forall$ $nbhd$ $N$ of p , $N$ has point in both $D$ and $D^{c}$ , so it follows by the definition of boundary that $l\in bd(D)$.

This means $bd(D)$ contains all its limits points.

 \noindent 
 $\therefore$ The boundary of a set closed.            \spaceskip1cm   \hfill                     $\square$
 

\end{document}
